\documentclass[a4paper,12pt]{article}

% Настройки языка и шрифтов
\usepackage[T2A]{fontenc}
\usepackage[utf8]{inputenc}
\usepackage[russian]{babel}
\usepackage{amsmath,amsfonts,amssymb}
\usepackage{graphicx}
\usepackage{color}
\usepackage{hyperref}
\usepackage{geometry}
\usepackage{indentfirst}
\usepackage{longtable}
\usepackage{booktabs}
\usepackage{float}

\geometry{left=3cm,right=1.5cm,top=2cm,bottom=2cm}

\begin{document}

% Титульный лист
\begin{titlepage}
    \centering
    {\fontsize{14pt}{16pt}\selectfont МОСКОВСКИЙ АВИАЦИОННЫЙ ИНСТИТУТ\\(НАЦИОНАЛЬНЫЙ ИССЛЕДОВАТЕЛЬСКИЙ УНИВЕРСИТЕТ)}\\[0.5cm]
    {\fontsize{12pt}{14pt}\selectfont Институт №8 «Компьютерные науки и прикладная математика»}\\[4cm]
    
    {\fontsize{16pt}{18pt}\selectfont \textbf{Лабораторные работы}}\\
    {\fontsize{14pt}{16pt}\selectfont \textbf{по курсу «Информационный поиск»}}\\[5cm]
    
    \vfill
    \begin{flushright}
        \begin{minipage}{0.51\textwidth}
            Выполнил: Жиденко Александр Сергеевич\\
            Группа: М8О-401Б-22\\
            Преподаватель: Кухтичев Антон Алексеевич
        \end{minipage}
    \end{flushright}
    
    \vfill
    {\large Москва, 2025}
\end{titlepage}

\tableofcontents
\newpage

\section{Введение}
В рамках курса «Информационный поиск» реализована полноценная учебная поисковая система на базе корпуса статей Lenta.ru. Все этапы — от добычи данных до индексации и булева поиска — выполнены на языке Python (Flask для веб-интерфейсов), без привлечения сторонних библиотек для основных структур данных индексации и поиска. Дополнительно подготовлены CLI-утилиты и веб-прототипы для демонстрации работы каждой лабораторной.

\section{Лабораторная №1: Добыча корпуса документов}
\subsection{Цель}
Собрать корпус не менее 30\,000 текстовых документов единой тематики для дальнейшей индексации.
\subsection{Источник и структура}
Использован архив новостей Lenta.ru. Каждый документ содержит заголовок, источник, URL, дату и основной текст в UTF-8.
\subsection{Реализация}
Написан скрипт-коллектор (Python, requests, BeautifulSoup) с повторными попытками, логированием и сохранением состояния. Документы сохраняются в пары \texttt{doc\_XXXXXX.txt} / \texttt{doc\_XXXXXX.meta.json}.
\subsection{Результаты}
Собрано 30000 документов для разработки.

\section{Лабораторная №2: Поисковый робот}
\subsection{Цель}
Создать веб-краулер с вежливостью, поддержкой \texttt{robots.txt}, очередью URL и сохранением статуса.
\subsection{Реализация}
Класс \texttt{WebCrawler} реализует:
\begin{itemize}
    \item разбор \texttt{robots.txt} и учёт \texttt{crawl-delay};
    \item нормализацию URL и дедупликацию;
    \item извлечение ссылок и текста (несколько CSS-селекторов + fallback по \texttt{body});
    \item сохранение документов в формат корпуса;
    \item CLI и веб-интерфейс (Flask) для запуска и мониторинга.
\end{itemize}
\subsection{Результаты}
Получен воспроизводимый краулер с логами, резюмированием и статистикой посещённых URL.

\section{Лабораторная №3: Токенизация}
\subsection{Цель}
Разбить текст на токены, нормализовать регистр, удалить пунктуацию и (опционально) стоп-слова.
\subsection{Реализация}
Класс \texttt{Tokenizer}:
\begin{itemize}
    \item regex для русских/английских слов и чисел;
    \item опции: \texttt{lowercase}, \texttt{remove\_punctuation}, \texttt{min\_length}, \texttt{remove\_stopwords};
    \item подсчёт частот и словаря;
    \item пакетная обработка корпуса с сохранением статистики и токенов.
\end{itemize}
\subsection{Результаты}
Подготовлены токены и частоты для дальнейших лабораторных; есть CLI и веб-демо.

\section{Лабораторная №4: Стемминг}
\subsection{Цель}
Привести токены к базовой форме с помощью стемминга.
\subsection{Реализация}
Класс \texttt{Stemmer} (русский — правилоудаление суффиксов; английский — упрощённый Porter-like):
\begin{itemize}
    \item обработка рефлексивных, прилагательных, глагольных и именных суффиксов;
    \item частоты основ, словарь основ, отображение токен $\rightarrow$ основа;
    \item пакетная обработка корпуса, CLI и веб-интерфейс.
\end{itemize}
\subsection{Результаты}
Уменьшено количество уникальных форм, подготовлены данные для индексации.

\section{Лабораторная №5: Закон Ципфа}
\subsection{Цель}
Проверить распределение частот токенов и соответствие закону Ципфа.
\subsection{Реализация}
Класс \texttt{ZipfAnalyzer}:
\begin{itemize}
    \item расчёт частот, рангов, константы $C$ и корреляции;
    \item графики log--log и rank--frequency (matplotlib, numpy);
    \item экспорт статистики и графиков, CLI и веб-интерфейс с встраиваемыми изображениями.
\end{itemize}
\subsection{Результаты}
Получено распределение частот, подтверждающее ожидаемую гиперболическую зависимость (корреляция для корпуса порядка 30\,000 документов ожидается $> 0{,}8$).

\section{Лабораторная №6: Булев индекс}
\subsection{Цель}
Построить булев индекс «термин $\rightarrow$ множество документов».
\subsection{Реализация}
Класс \texttt{BooleanIndex}:
\begin{itemize}
    \item построение из корпуса с токенизацией и (опц.) стеммингом;
    \item сериализация/загрузка в JSON, текстовый экспорт;
    \item статистика: объём, топ-термины, среднее число термов на документ.
\end{itemize}
\subsection{Результаты}
Индекс готов для булева поиска; предусмотрены CLI и веб-обвязка.

\section{Лабораторная №7: Булев поиск}
\subsection{Цель}
Реализовать обработку булевых запросов (AND, OR, NOT) на основе индекса.
\subsection{Реализация}
Класс \texttt{BooleanSearch}:
\begin{itemize}
    \item разбор запроса в постфиксную нотацию с приоритетом NOT $>$ AND $>$ OR;
    \item оценка через операции над множествами документов;
    \item выдача с метаданными, лимитирование результатов;
    \item CLI (одиночный, пакетный, интерактивный) и веб-интерфейс (Flask).
\end{itemize}
\subsection{Результаты}
Полностью функциональный булев поиск по корпусу. Запросы с несколькими операторами и скобками поддерживаются.

\section{Заключение}
Реализован полный цикл учебной поисковой системы: сбор данных, подготовка текста, стемминг, статистический анализ, построение индекса и булев поиск. Подготовлены CLI-утилиты, веб-интерфейсы и автотесты для каждого этапа. Код и структура данных позволяют масштабировать корпус до 30\,000+ документов, сохраняя совместимость с отчётными требованиями курса.

\end{document}
